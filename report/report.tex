\documentclass[acmlarge]{acmart}

\AtBeginDocument{%
  \providecommand\BibTeX{{%
    Bib\TeX}}}

\begin{document}

\title{On the parallelization of a N-body simulation}

\author{Alice}
\email{alice@mail.com}
\affiliation{
  \institution{Universidade do Minho}
  \city{Braga}
  \country{Portugal}
}
\author{Bob}
\email{bob@mail.com}
\affiliation{
  \institution{Universidade do Minho}
  \city{Braga}
  \country{Portugal}
}
\author{Charlie}
\email{charlie@mail.com}
\affiliation{
  \institution{Universidade do Minho}
  \city{Braga}
  \country{Portugal}
}

\renewcommand{\shortauthors}{Alice et al.}


\begin{abstract}
  This report documents the parallelization techniques implemented on a N-body simulation. This was achieved through the OpenMP API. After carefully analyzing and tested different optimization locations, the resulting program had a xxx\% speedup compared to the original, sequential implementation. This result shows that physical, N-body simulations are an excellent target for parallel code optimizations. As even very simple optimization techniques are able to produce a huge speedup.
\end{abstract}

\keywords{Parallel, Parallelization, Code Optimization, Simulation, Physics Simulation, OpenMP}


\maketitle

\section{Introduction}

With the gradual decline of Moore's Law, the time of swapping an older chip for a newer one and experiencing exponential speedups is coming to an end. This, in a sense, limits the computing power a certain problem can have to still be tractable. That is, if we were dealing with a single computing core on a machine.

Fortunatly, that is not the case. Nowadays, even the simplest computers have multiple computing cores. Unfortunatly, this parallelism is often misused or even neglected, making the device, essentially, a single-core computer.



\end{document}
\endinput
